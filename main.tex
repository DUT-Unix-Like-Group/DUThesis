%!TeX program = xelatex
% Note: This line can't be removed if you are using latexmk as compile

% Compile with: xelatex -> biber -> xelatex -> xelatex

% Author: JC Zhang
% Acknowledgement: BIT

% DUT!

\documentclass[UTF8,AutoFakeBold=3,AutoFakeSlant,zihao=-4]{ctexart}
\usepackage[a4paper,left=2.5cm,right=2.5cm,top=2.5cm,bottom=2.5cm,includeheadfoot]{geometry}
\usepackage{fontspec}
\usepackage{setspace}
\usepackage{graphicx}
\usepackage{fancyhdr}
\usepackage{pdfpages}
\usepackage{setspace}
\usepackage{booktabs}
\usepackage{multirow}
\usepackage{caption}
\usepackage{amsmath}
\usepackage{amsfonts}
\usepackage{amssymb}
\usepackage{float}
\usepackage{tabularx}

% 设置参考文献编译后端为 biber,引用格式为 GB/T7714-2015 格式
% 参考文献使用宏包见 https://github.com/hushidong/biblatex-gb7714-2015
\usepackage{biblatex}
% \usepackage[style=gb7714-2015,backend=biber]{biblatex}

% 参考文献引用文件 refs.bib
\addbibresource{misc/refs.bib}

\newcommand{\id}{}
\newcommand{\name}{}
\newcommand{\level}{}
\newcommand{\reposName}{}
\newcommand{\type}{}
\newcommand{\mentor}{}
\newcommand{\dept}{}
\newcommand{\datenow}{}

% % 定义 caption 字体为楷体
% \DeclareCaptionFont{kaiticaption}{\kaishu \normalsize}
%
% % 设置图片的 caption 格式
% \renewcommand{\thefigure}{\thesection-\arabic{figure}}
% \captionsetup[figure]{font=small,labelsep=space,skip=10bp,labelfont=bf,font=kaiticaption}
%
% % 设置表格的 caption 格式
% \renewcommand{\thetable}{\thesection-\arabic{table}}
% \captionsetup[table]{font=small,labelsep=space,skip=10bp,labelfont=bf,font=kaiticaption}


% \CTEXoptions[today=small]

% 将西文字体设置为 Times New Roman
\setromanfont{Times New Roman}

%% 将中文楷体设置为 SIMKAI.TTF(如果需要)
% \setCJKfamilyfont{zhkai}{[SIMKAI.TTF]}
% \newcommand*{\kaiti}{\CJKfamily{zhkai}}

% 设置文档标题深度
\setcounter{tocdepth}{3}
\setcounter{secnumdepth}{3}

%%
% 设置一级标题、二级标题格式
% 一级标题:小三,宋体,加粗,段前段后各半行
\ctexset{section={
  format={\raggedright \bfseries \songti \zihao{-3}},
  beforeskip = 24bp plus 1ex minus .2ex,
  afterskip = 24bp plus .2ex,
  fixskip = true,
  number={\chinese{section}},
  name = {,.}
  }
}
% 二级标题:小四,宋体,加粗,段前段后各半行
\ctexset{subsection={
  format = {\songti \raggedright \zihao{4}},
  beforeskip =24bp plus 1ex minus .2ex,
  afterskip = 24bp plus .2ex,
  fixskip = true,
  name={(,)},
  number={\chinese{subsection}}
  }
}

% \CTEXsetup[name={,、}, number={\chinese{section}}]{section}
% \CTEXsetup[name={(,)}, number={\chinese{subsection}}]{subsection}

\begin{document}



\topskip=0pt

\setCJKfamilyfont{hwkt}{STKaiti}
\newcommand{\hwk}{\CJKfamily{hwkt}}


\begin{titlepage}
	\vspace*{30mm}
	\centering
	\hspace{-6mm}\hwk\fontsize{24pt}{24pt}\selectfont{\textbf{大连理工大学大学生创新创业训练计划}}

	\vspace{13mm}

	\hspace{-6mm}\heiti\fontsize{26pt}{26pt}\selectfont{\textbf{项目开题报告}}

	\vspace{33mm}

	\flushleft
	\begin{spacing}{1.6}
		\hspace{15mm}\songti\fontsize{14pt}{14pt}\selectfont{项\hspace{1em}目\hspace{1em}编\hspace{1em}号:}\underline{\makebox[70mm][c]{\id}}

		\hspace{15mm}\songti\fontsize{14pt}{14pt}\selectfont{项\hspace{1em}目\hspace{1em}名\hspace{1em}称:}\underline{\makebox[70mm][c]{\name}}

		\hspace{15mm}\songti\fontsize{14pt}{14pt}\selectfont{项\hspace{1em}目\hspace{1em}级\hspace{1em}别:}\underline{\makebox[70mm][c]{\level}}

		\hspace{15mm}\songti\fontsize{14pt}{14pt}\selectfont{项\hspace{2mm}目\hspace{2mm}负\hspace{2mm}责\hspace{2mm}人:}\underline{\makebox[72mm][c]{\reposName}}

		\hspace{15mm}\songti\fontsize{14pt}{14pt}\selectfont{项\hspace{1em}目\hspace{1em}类\hspace{1em}型:}\underline{\makebox[70mm][c]{
				\begin{tabular}{cc}
					\makebox[0pt][l]{$\square$}\raisebox{.15ex}{\hspace{0.1em}$\checkmark$} 创新训练 & $\square$ 科研训练 \\
					$\square$ 创业训练                                                               & $\square$ 创业实践 \\
				\end{tabular}
			}}

		\vspace{2mm}

		\hspace{15mm}\songti\fontsize{14pt}{14pt}\selectfont{指\hspace{1em}导\hspace{1em}老\hspace{1em}师:}\underline{\makebox[70mm][c]{\mentor}}

		\hspace{15mm}\songti\fontsize{14pt}{14pt}\selectfont{所\hspace{1.7mm}在\hspace{1.7mm}学\hspace{1.7mm}部\hspace{1.7mm}学\hspace{1.7mm}院:}\underline{\makebox[67mm][c]{\dept}}

	\end{spacing}

	\vspace{40mm}

	\centering
	\songti\fontsize{18pt}{18pt}\selectfont{\textbf{大\hspace{1em}连\hspace{1em}理\hspace{1em}工\hspace{1em}大\hspace{1em}学}}

	\vspace{10mm}

	\songti\fontsize{12pt}{12pt}\selectfont{\textbf{\datenow}}
\end{titlepage}



\pagestyle{fancy}
%
% \fancyhf{}
% \fancyhead[R]{\fontsize{10.5pt}{10.5pt}\selectfont{大连理工大学大学生创新创业训练计划}}
% \fancyfoot[R]{\fontsize{9pt}{9pt}\selectfont{\thepage}}
% \renewcommand{\headrulewidth}{1pt}
% \renewcommand{\footrulewidth}{0pt}

\fancyhead{}
\fancyfoot{}

% 正文 20 磅的行距,段前段后间距为 0
\setlength{\baselineskip}{20pt}
% 正文首行悬挂 1.02cm
\setlength{\parindent}{2em}

\section{研究动机}
\section{参考文献}
\printbibliography[heading=none]

\end{document}
